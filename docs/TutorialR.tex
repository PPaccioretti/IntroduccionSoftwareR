% Options for packages loaded elsewhere
\PassOptionsToPackage{unicode}{hyperref}
\PassOptionsToPackage{hyphens}{url}
%
\documentclass[
]{book}
\usepackage{amsmath,amssymb}
\usepackage{lmodern}
\usepackage{iftex}
\ifPDFTeX
  \usepackage[T1]{fontenc}
  \usepackage[utf8]{inputenc}
  \usepackage{textcomp} % provide euro and other symbols
\else % if luatex or xetex
  \usepackage{unicode-math}
  \defaultfontfeatures{Scale=MatchLowercase}
  \defaultfontfeatures[\rmfamily]{Ligatures=TeX,Scale=1}
\fi
% Use upquote if available, for straight quotes in verbatim environments
\IfFileExists{upquote.sty}{\usepackage{upquote}}{}
\IfFileExists{microtype.sty}{% use microtype if available
  \usepackage[]{microtype}
  \UseMicrotypeSet[protrusion]{basicmath} % disable protrusion for tt fonts
}{}
\makeatletter
\@ifundefined{KOMAClassName}{% if non-KOMA class
  \IfFileExists{parskip.sty}{%
    \usepackage{parskip}
  }{% else
    \setlength{\parindent}{0pt}
    \setlength{\parskip}{6pt plus 2pt minus 1pt}}
}{% if KOMA class
  \KOMAoptions{parskip=half}}
\makeatother
\usepackage{xcolor}
\IfFileExists{xurl.sty}{\usepackage{xurl}}{} % add URL line breaks if available
\IfFileExists{bookmark.sty}{\usepackage{bookmark}}{\usepackage{hyperref}}
\hypersetup{
  pdftitle={Introducción al software R},
  pdfauthor={Paccioretti Pablo; Bruno Cecilia; González Montoro Aldana; Nores María Laura},
  hidelinks,
  pdfcreator={LaTeX via pandoc}}
\urlstyle{same} % disable monospaced font for URLs
\usepackage{color}
\usepackage{fancyvrb}
\newcommand{\VerbBar}{|}
\newcommand{\VERB}{\Verb[commandchars=\\\{\}]}
\DefineVerbatimEnvironment{Highlighting}{Verbatim}{commandchars=\\\{\}}
% Add ',fontsize=\small' for more characters per line
\usepackage{framed}
\definecolor{shadecolor}{RGB}{248,248,248}
\newenvironment{Shaded}{\begin{snugshade}}{\end{snugshade}}
\newcommand{\AlertTok}[1]{\textcolor[rgb]{0.94,0.16,0.16}{#1}}
\newcommand{\AnnotationTok}[1]{\textcolor[rgb]{0.56,0.35,0.01}{\textbf{\textit{#1}}}}
\newcommand{\AttributeTok}[1]{\textcolor[rgb]{0.77,0.63,0.00}{#1}}
\newcommand{\BaseNTok}[1]{\textcolor[rgb]{0.00,0.00,0.81}{#1}}
\newcommand{\BuiltInTok}[1]{#1}
\newcommand{\CharTok}[1]{\textcolor[rgb]{0.31,0.60,0.02}{#1}}
\newcommand{\CommentTok}[1]{\textcolor[rgb]{0.56,0.35,0.01}{\textit{#1}}}
\newcommand{\CommentVarTok}[1]{\textcolor[rgb]{0.56,0.35,0.01}{\textbf{\textit{#1}}}}
\newcommand{\ConstantTok}[1]{\textcolor[rgb]{0.00,0.00,0.00}{#1}}
\newcommand{\ControlFlowTok}[1]{\textcolor[rgb]{0.13,0.29,0.53}{\textbf{#1}}}
\newcommand{\DataTypeTok}[1]{\textcolor[rgb]{0.13,0.29,0.53}{#1}}
\newcommand{\DecValTok}[1]{\textcolor[rgb]{0.00,0.00,0.81}{#1}}
\newcommand{\DocumentationTok}[1]{\textcolor[rgb]{0.56,0.35,0.01}{\textbf{\textit{#1}}}}
\newcommand{\ErrorTok}[1]{\textcolor[rgb]{0.64,0.00,0.00}{\textbf{#1}}}
\newcommand{\ExtensionTok}[1]{#1}
\newcommand{\FloatTok}[1]{\textcolor[rgb]{0.00,0.00,0.81}{#1}}
\newcommand{\FunctionTok}[1]{\textcolor[rgb]{0.00,0.00,0.00}{#1}}
\newcommand{\ImportTok}[1]{#1}
\newcommand{\InformationTok}[1]{\textcolor[rgb]{0.56,0.35,0.01}{\textbf{\textit{#1}}}}
\newcommand{\KeywordTok}[1]{\textcolor[rgb]{0.13,0.29,0.53}{\textbf{#1}}}
\newcommand{\NormalTok}[1]{#1}
\newcommand{\OperatorTok}[1]{\textcolor[rgb]{0.81,0.36,0.00}{\textbf{#1}}}
\newcommand{\OtherTok}[1]{\textcolor[rgb]{0.56,0.35,0.01}{#1}}
\newcommand{\PreprocessorTok}[1]{\textcolor[rgb]{0.56,0.35,0.01}{\textit{#1}}}
\newcommand{\RegionMarkerTok}[1]{#1}
\newcommand{\SpecialCharTok}[1]{\textcolor[rgb]{0.00,0.00,0.00}{#1}}
\newcommand{\SpecialStringTok}[1]{\textcolor[rgb]{0.31,0.60,0.02}{#1}}
\newcommand{\StringTok}[1]{\textcolor[rgb]{0.31,0.60,0.02}{#1}}
\newcommand{\VariableTok}[1]{\textcolor[rgb]{0.00,0.00,0.00}{#1}}
\newcommand{\VerbatimStringTok}[1]{\textcolor[rgb]{0.31,0.60,0.02}{#1}}
\newcommand{\WarningTok}[1]{\textcolor[rgb]{0.56,0.35,0.01}{\textbf{\textit{#1}}}}
\usepackage{longtable,booktabs,array}
\usepackage{calc} % for calculating minipage widths
% Correct order of tables after \paragraph or \subparagraph
\usepackage{etoolbox}
\makeatletter
\patchcmd\longtable{\par}{\if@noskipsec\mbox{}\fi\par}{}{}
\makeatother
% Allow footnotes in longtable head/foot
\IfFileExists{footnotehyper.sty}{\usepackage{footnotehyper}}{\usepackage{footnote}}
\makesavenoteenv{longtable}
\usepackage{graphicx}
\makeatletter
\def\maxwidth{\ifdim\Gin@nat@width>\linewidth\linewidth\else\Gin@nat@width\fi}
\def\maxheight{\ifdim\Gin@nat@height>\textheight\textheight\else\Gin@nat@height\fi}
\makeatother
% Scale images if necessary, so that they will not overflow the page
% margins by default, and it is still possible to overwrite the defaults
% using explicit options in \includegraphics[width, height, ...]{}
\setkeys{Gin}{width=\maxwidth,height=\maxheight,keepaspectratio}
% Set default figure placement to htbp
\makeatletter
\def\fps@figure{htbp}
\makeatother
\setlength{\emergencystretch}{3em} % prevent overfull lines
\providecommand{\tightlist}{%
  \setlength{\itemsep}{0pt}\setlength{\parskip}{0pt}}
\setcounter{secnumdepth}{5}
\usepackage[utf8]{inputenc}
\usepackage{graphicx}
\usepackage{booktabs}
\usepackage{longtable}
\usepackage{framed,color}
\definecolor{shadecolor}{RGB}{248,248,248}
\usepackage{polyglossia}
\setmainlanguage{spanish}
\AtBeginDocument{\renewcommand{\chaptername}{Sección}}
\usepackage{longtable}


\usepackage{letltxmacro}
\LetLtxMacro\SavedIncludeGraphics\includegraphics
\def\includegraphics#1#{% #1 catches optional stuff (star/opt. arg.)
	\IncludeGraphicsAux{#1}%
}%
\newcommand*{\IncludeGraphicsAux}[2]{%
	\XeTeXLinkBox{%
		\SavedIncludeGraphics#1{#2}%
	}%
}%

\newenvironment{rmdblock}[1]
{\begin{shaded*}
		\begin{itemize}
			\renewcommand{\labelitemi}{
				\raisebox{-.7\height}[0pt][0pt]{
					{\setkeys{Gin}{width=3em,keepaspectratio}\includegraphics{images/#1}}
				}
			}
			\item
		}
		{
		\end{itemize}
	\end{shaded*}
}
\newenvironment{rmdnote}
{\begin{rmdblock}{note}}
	{\end{rmdblock}}
\newenvironment{rmdcaution}
{\begin{rmdblock}{caution}}
	{\end{rmdblock}}
\newenvironment{rmdimportant}
{\begin{rmdblock}{important}}
	{\end{rmdblock}}
\newenvironment{rmdtip}
{\begin{rmdblock}{tip}}
	{\end{rmdblock}}
\newenvironment{rmdwarning}
{\begin{rmdblock}{warning}}
	{\end{rmdblock}}

\usepackage{environ}
\usepackage{xcolor}
\NewEnviron{boxeda}%
  {\begin{center}%
   \noindent\fcolorbox{black}{gray!30}%
     {\parbox{33em}%
       {\vspace{15pt}\par
        \BODY
        \vspace{15pt}\par
       }%
     }%
   \end{center}%
  }
  
% \newenvironment{boxeda}
%     {\begin{center}
%       \begin{tabular}{|p{0.9\textwidth}|}
%        %\hline\\
%     }
%     { 
%     %\\\\\hline
%     \end{tabular} 
%     \end{center}
%     }

% \usepackage{booktabs}
% \usepackage{longtable}
% \usepackage{framed,color}
% \definecolor{shadecolor}{RGB}{248,248,248}
% 
% \ifxetex
%   \usepackage{letltxmacro}
%   \setlength{\XeTeXLinkMargin}{1pt}
%   \LetLtxMacro\SavedIncludeGraphics\includegraphics
%   \def\includegraphics#1#{% #1 catches optional stuff (star/opt. arg.)
%     \IncludeGraphicsAux{#1}%
%   }%
%   \newcommand*{\IncludeGraphicsAux}[2]{%
%     \XeTeXLinkBox{%
%       \SavedIncludeGraphics#1{#2}%
%     }%
%   }%
% \fi
% 
% \newenvironment{rmdblock}[1]
%   {\begin{shaded*}
%   \begin{itemize}
%   \renewcommand{\labelitemi}{
%     \raisebox{-.7\height}[0pt][0pt]{
%       {\setkeys{Gin}{width=3em,keepaspectratio}\includegraphics{images/#1}}
%     }
%   }
%   \item
%   }
%   {
%   \end{itemize}
%   \end{shaded*}
%   }
% \newenvironment{rmdnote}
%   {\begin{rmdblock}{note}}
%   {\end{rmdblock}}
% \newenvironment{rmdcaution}
%   {\begin{rmdblock}{caution}}
%   {\end{rmdblock}}
% \newenvironment{rmdimportant}
%   {\begin{rmdblock}{important}}
%   {\end{rmdblock}}
% \newenvironment{rmdtip}
%   {\begin{rmdblock}{tip}}
%   {\end{rmdblock}}
% \newenvironment{rmdwarning}
%   {\begin{rmdblock}{warning}}
%   {\end{rmdblock}}
\ifLuaTeX
  \usepackage{selnolig}  % disable illegal ligatures
\fi
\usepackage[]{natbib}
\bibliographystyle{apalike}

\title{Introducción al software R}
\usepackage{etoolbox}
\makeatletter
\providecommand{\subtitle}[1]{% add subtitle to \maketitle
  \apptocmd{\@title}{\par {\large #1 \par}}{}{}
}
\makeatother
\subtitle{Notas de clases}
\author{Paccioretti Pablo\footnote{\href{mailto:pablopaccioretti@agro.unc.edu.ar}{\nolinkurl{pablopaccioretti@agro.unc.edu.ar}}} \and Bruno Cecilia\footnote{\href{mailto:cebruno@agro.unc.edu.ar}{\nolinkurl{cebruno@agro.unc.edu.ar}}} \and González Montoro Aldana\footnote{\href{mailto:aldana.goonzalez.montoro@unc.edu.ar}{\nolinkurl{aldana.goonzalez.montoro@unc.edu.ar}}} \and Nores María Laura\footnote{\href{mailto:lalinores@yahoo.com.ar}{\nolinkurl{lalinores@yahoo.com.ar}}}}
\date{}

\begin{document}
\maketitle

{
\setcounter{tocdepth}{1}
\tableofcontents
}
\hypertarget{instalaciuxf3n-de-programas}{%
\chapter{Instalación de programas}\label{instalaciuxf3n-de-programas}}

R puede ser instalado en múltiples plataformas tales como Windows, MacOS y en
sistemas basados en Linux. Además hay múltiples entornos de desarrollo
integrado (\emph{Integrated Development Environment} IDE) los cuales facilitan la
programación. Ejemplos de este tipo de software es RStudio \citep{Rstudio} y el
intérprete de R que contiene InfoStat \citep{Infostat}. Las interfaces gráficas
de ambos software son similares.

Links para las descargas:

\begin{itemize}
\tightlist
\item
  \href{https://cran.r-project.org/bin/windows/base/}{R (windows)}
\item
  \href{https://www.rstudio.com/products/rstudio/download/\#download}{RStudio}
\item
  \href{http://www.infostat.com.ar/descargas/demo/infostatinstaller_esp.exe}{InfoStat}
\end{itemize}

\hypertarget{interfaz-de-rstudio}{%
\section{Interfaz de RStudio}\label{interfaz-de-rstudio}}

La interfaz de RStudio se divide en cuatro paneles.

\begin{figure}
\centering
\includegraphics[width=0.8\textwidth,height=\textheight]{images/Rstudiopartes.jpg}
\caption{Diseño de los paneles de RStudio}
\end{figure}

El panel superior izquierdo permite al usuario editar scripts (líneas de código),
para esto puede cargar scripts previamente escritos o escribir nuevos. En el
panel de abajo a la izquierda (\emph{consola}) se muestran las sentencias de código
ejecutadas y los resultados. En este panel el software se ``comunica'' con el usuario.
En los paneles derechos se muestran los objetos cargados en el ambiente de
trabajo, mientras que en el panel inferior derecho se muestran principalmente los
archivos del directorio de trabajo, gráficos generados durante la sesión y ayudas
de funciones específicas.

\hypertarget{interfaz-del-intuxe9rprete-de-infostat}{%
\section{Interfaz del intérprete de InfoStat}\label{interfaz-del-intuxe9rprete-de-infostat}}

La interfaz del intérprete de R en InfoStat se divide en cuatro paneles.

\begin{figure}
\centering
\includegraphics[width=0.8\textwidth,height=\textheight]{images/InfoStatPartes.png}
\caption{Diseño de los paneles del intérprete de R de InfoStat RStudio}
\end{figure}

El panel superior izquierdo permite al usuario visualizar o editar scripts.
En el panel inferior izquierdo se muestran los resultados de funciones.
En los paneles derechos se muestran los objetos cargados en el ambiente de
trabajo, mientras que en el panel inferior derecho se muestran los paquetes
instalados y en rojo los paquetes cargados en memoria.

\hypertarget{intro}{%
\chapter{El lenguaje R}\label{intro}}

R \citep{R-base} es un lenguaje de programación orientado a objetos. Fue creado por
Ross Ihaka y Robert Gentleman en 1993 como un dialecto del software S, fue
publicado en 1996 \citep{ihaka1996r}. Es un software libre y de código abierto,
lo que significa que se puede usar, compartir y modificar el software libremente.
Junto con el instalador de R, se distribuyen ciertos paquetes (\emph{packages}) los
cuales incluyen funciones para implementar algunos métodos estadísticos
clásicos y modernos. Pot esta razón, muchas personas utilizan R para realizar análisis
estadísticos. Muchos algoritmos y metodologías estadísticas están disponibles
para ser implementadas en R, pero muchas de ellas se encuentran en paquetes
específicos que no se encuentran en la instalación básica de R, por lo que para
poder utilizar estas funciones, se debe descargar el paquete.

\hypertarget{SintaxisBasica}{%
\section{Generalidades del ambiente R}\label{SintaxisBasica}}

R distingue mayúsculas y minúsculas, esto significa que \texttt{Producto} y \texttt{producto}
son dos palabras diferentes. Los comandos elementales consisten en expresiones
o asignaciones. Si se ejecuta una expresión el resultado se imprimirá en la
consola pero no se guardará dicho valor. Cuando se asigna un valor de una
expresión (mediante el comando \texttt{\textless{}-}), el resultado no se imprimirá en pantalla
pero el valor será asignado a un objeto que luego podrá reutilizarse.
Diferentes sentencias de código (comandos) pueden ser separados por \texttt{;} o por
una nueva línea dentro del script, la segunda opción es la más utilizada por los
usuarios. Un conjunto de comandos pueden estar encerrados entre llaves (\texttt{\{}
y \texttt{\}}). Los \texttt{\#} indican comentarios en el código, todo lo que está a la
derecha de este símbolo no será ejecutado por R. Si se desean hacer comentarios
de más de una línea, cada una de ellas debe comenzar con \texttt{\#}.

Si deseamos asignar en el objeto llamado \texttt{x} el valor del resultado de aplicar
la raíz cuadrada al número 10, debemos utilizar la función \texttt{\textless{}-}:

\begin{Shaded}
\begin{Highlighting}[]
\NormalTok{x }\OtherTok{\textless{}{-}} \FunctionTok{sqrt}\NormalTok{(}\DecValTok{10}\NormalTok{) }\CommentTok{\#No se muestra el resultado}
\end{Highlighting}
\end{Shaded}

Para ver el valor de cualquier objeto, se puede especificar el nombre y ejecutar la línea, por ejemplo si deseamos ver el valor que está almacenado en \texttt{x} debemos escribir y ejecutar:

\begin{Shaded}
\begin{Highlighting}[]
\NormalTok{x }\CommentTok{\#Se muestra el resultado}
\end{Highlighting}
\end{Shaded}

\begin{verbatim}
## [1] 3.162278
\end{verbatim}

\begin{Shaded}
\begin{Highlighting}[]
\FunctionTok{sqrt}\NormalTok{(}\DecValTok{10}\NormalTok{) }\CommentTok{\#Se imprime en la consola el resultado}
\end{Highlighting}
\end{Shaded}

\begin{verbatim}
## [1] 3.162278
\end{verbatim}

Las funciones son segmentos de código escrito para llevar a cabo una tarea
específica, en el ejemplo anterior se utilizó la función \texttt{sqrt} para calcular
la raíz cuadrada de 10. Las funciones pueden necesitar argumentos y devuelven
uno o más valores en el resultado, algunas funciones pueden no devolver ningún
resultado visible. Los argumentos de la función son los \emph{inputs} para ejecutar
la tarea. Argumentos deben ir dentro de paréntesis luego del nombre de la función,
cada argumento se separa con \texttt{,} (\texttt{(arg1,\ arg2\ )}). Nombres de los argumentos
pueden especificarse explícitamente o no, en el ejemplo anterior los argumentos no
se especificaron con nombre explícito. Si no se detalla el nombre del argumento,
R entenderá que están en el mismo orden que se especificaron cuando se creó la
función. En el caso de \texttt{sqrt} el primer y único argumento de la función es un
objeto numérico.

\begin{rmdnote}
Notar que los nombre de la mayoría de las funciones de R derivan del
inglés y que R utiliza \texttt{.} como separador decimal.
\end{rmdnote}

\hypertarget{funciones-y-comandos-buxe1sicos}{%
\section{Funciones y comandos básicos}\label{funciones-y-comandos-buxe1sicos}}

En R se puede ejecutar cualquier operación matemática. Comencemos viendo
algunas operaciones básicas:

Suma:

\begin{Shaded}
\begin{Highlighting}[]
\DecValTok{5} \SpecialCharTok{+} \DecValTok{2}
\end{Highlighting}
\end{Shaded}

\begin{verbatim}
## [1] 7
\end{verbatim}

Raíz cuadrada:

\begin{Shaded}
\begin{Highlighting}[]
\FunctionTok{sqrt}\NormalTok{(}\DecValTok{15}\NormalTok{)}
\end{Highlighting}
\end{Shaded}

\begin{verbatim}
## [1] 3.872983
\end{verbatim}

\hypertarget{tablas_resumen}{%
\subsection{Tablas resumen de operadores y funciones}\label{tablas_resumen}}

\begin{longtable}[t]{ll}
\caption{\label{tab:funcbasic}Algunas funciones matemáticas en R}\\
\toprule
Sintaxis & Operación\\
\midrule
`x + y` & suma de x e y\\
`x - y` & diferencia de x e y\\
`x * y` & multiplicación de x e y\\
`x / y` & división de x por y\\
`x \%/\% y` & parte entera de la división de x por y\\
\addlinespace
`x \%\% y` & resto de la división de x por y\\
`x \textasciicircum{} y` & x elevado a y-ésima potencia\\
`x < y` & x menor que y\\
`x <= y` & x menor o igual que y\\
`x > y` & x mayor que y\\
\addlinespace
`x >= y` & x mayor o igual que y\\
`x == y` & x igual a y\\
`x != y` & x no es igual a y\\
`sqrt(x)` & raíz cuadrada de x\\
`exp(x)` & exponencial de x\\
\addlinespace
`log(x)` & logaritmo natural de x\\
`log(x, k)` & logaritmo base k de x\\
`sum(x)` & suma de los elementos de x\\
`prod(x)` & producto de los elementos de x\\
`round(x, k)` & x redondeado a k dígitos\\
\bottomrule
\end{longtable}

\hypertarget{ayuda}{%
\subsection{Ayuda}\label{ayuda}}

R incluye documentación de ayuda muy detallada. Para acceder a la ayuda de cada
función, objeto o datos de prueba se debe ejecutar el comando \texttt{help()} o \texttt{?}.
Por ejemplo \texttt{help(sqrt)}, o \texttt{?sqrt}. Otra forma de pedir la ayuda es
presionando la tecla F1 luego de seleccionar el nombre de la función en RStudio.
La sentencia \texttt{??} busca un patrón dentro de la documentación del sistema de ayuda,
es útil si no se conoce cuál función ejecuta cierto análisis. Otra herramienta
muy útil para buscar ayuda es Google o \href{https://stackoverflow.com/}{Stack Overflow}.

\begin{Shaded}
\begin{Highlighting}[]
\FunctionTok{help}\NormalTok{(sqrt)}
\NormalTok{??square}
\end{Highlighting}
\end{Shaded}

\hypertarget{asignaciones}{%
\subsection{Asignaciones}\label{asignaciones}}

Como ya se especificó en la sección \ref{SintaxisBasica}, un comando de
asignación es \texttt{\textless{}-}, donde a la izquierda se especifica el nombre del objeto y a
la derecha el valor, ya sean resultados de un cálculo matemático o de un análisis
estadístico más complejo. Por ejemplo, si se desea asignar el valor de \texttt{5} al
objeto \texttt{radio} se debe ejecutar \texttt{radio\ \textless{}-\ 5}. Otras formas de hacer asignaciones
es mediante la utilización de \texttt{=} o \texttt{-\textgreater{}}, este último no es muy utilizado entre
los usuarios de R.

Asignaremos al objeto \texttt{x} una secuencia numérica del 1 al 5 y luego imprimiremos
el contenido de \texttt{x} en la consola:

\begin{Shaded}
\begin{Highlighting}[]
\NormalTok{x }\OtherTok{\textless{}{-}} \FunctionTok{c}\NormalTok{(}\DecValTok{1}\NormalTok{, }\DecValTok{2}\NormalTok{, }\DecValTok{3}\NormalTok{, }\DecValTok{4}\NormalTok{, }\DecValTok{5}\NormalTok{)  }\CommentTok{\#No se muestra el resultado}
\NormalTok{x                      }\CommentTok{\#Se auto imprime el resultado}
\DocumentationTok{\#\# [1] 1 2 3 4 5}
\FunctionTok{print}\NormalTok{(x)               }\CommentTok{\#Imprime el resultado de manera explícita }
\DocumentationTok{\#\# [1] 1 2 3 4 5}
                       \CommentTok{\#mediante el comando print }
\end{Highlighting}
\end{Shaded}

Otras formas de asignar valores es utilizando \texttt{-\textgreater{}} o \texttt{=}

\begin{Shaded}
\begin{Highlighting}[]
\FunctionTok{c}\NormalTok{(}\DecValTok{1}\NormalTok{, }\DecValTok{2}\NormalTok{, }\DecValTok{3}\NormalTok{) }\OtherTok{{-}\textgreater{}}\NormalTok{ x}
\NormalTok{x}
\end{Highlighting}
\end{Shaded}

\begin{verbatim}
## [1] 1 2 3
\end{verbatim}

\begin{Shaded}
\begin{Highlighting}[]
\NormalTok{x }\OtherTok{=} \FunctionTok{c}\NormalTok{(}\DecValTok{1}\NormalTok{, }\DecValTok{2}\NormalTok{, }\DecValTok{3}\NormalTok{, }\DecValTok{4}\NormalTok{)}
\NormalTok{x}
\end{Highlighting}
\end{Shaded}

\begin{verbatim}
## [1] 1 2 3 4
\end{verbatim}

\begin{rmdnote}
Al utilizar el comando de asignación con el mismo nombre de objeto
(\texttt{x}), cada vez que se utilizó ese comando, el valor que contenía
previamente se reasignó con el valor nuevo.
\end{rmdnote}

\hypertarget{r-como-herramienta-estaduxedstica}{%
\subsection{R como herramienta estadística}\label{r-como-herramienta-estaduxedstica}}

En el paquete \texttt{stats} (uno de los paquetes instalados por defecto al momento de
instala R) permite entre otras cosas, obtener la densidad, función de
distribución (probabilidades), cuantiles y generar números aleatorios de las
distribuciones estadísticas más comunes. Por ejemplo, si se desea generar 40
números aleatorios de una distribución normal estándar se deberá ejecutar la
sentencia \texttt{rnorm(40)}.

Si se desea calcular medidas descriptivas básicas de un vector se puede
ejecutar \texttt{mean} para calcular la media, \texttt{sd} para calcular el desvío estándar
y \texttt{var} para la varianza. Otra función útil para obtener valores de posición
es la función \texttt{summary}.

\begin{Shaded}
\begin{Highlighting}[]
\NormalTok{x }\OtherTok{\textless{}{-}} \FunctionTok{rnorm}\NormalTok{(}\DecValTok{40}\NormalTok{)}
\FunctionTok{summary}\NormalTok{(x)}
\end{Highlighting}
\end{Shaded}

\begin{verbatim}
##     Min.  1st Qu.   Median     Mean  3rd Qu.     Max. 
## -2.90063 -0.69488  0.02006 -0.07096  0.67489  1.21151
\end{verbatim}

\hypertarget{r-como-herramienta-gruxe1fica}{%
\subsection{R como herramienta gráfica}\label{r-como-herramienta-gruxe1fica}}

Con R se puede realizar gráficos y modificar numerosos parámetros de éste para
que pueda ser publicado. A modo de ejemplo se realizará un histograma y un
boxplot de la variable \texttt{x} generada anteriormente.

\begin{Shaded}
\begin{Highlighting}[]
\FunctionTok{hist}\NormalTok{(x)}
\end{Highlighting}
\end{Shaded}

\includegraphics{TutorialR_files/figure-latex/unnamed-chunk-14-1.pdf}

\begin{Shaded}
\begin{Highlighting}[]
\FunctionTok{boxplot}\NormalTok{(x)}
\end{Highlighting}
\end{Shaded}

\includegraphics{TutorialR_files/figure-latex/unnamed-chunk-14-2.pdf}

Podría decirse que la función más importante para generar gráficos dentro de
los paquetes instalados por defecto es \texttt{plot}. Esta función permite permite
realizar diagramas de dispersión y editar algunos elementos del gráfico.

\begin{Shaded}
\begin{Highlighting}[]
\NormalTok{x }\OtherTok{\textless{}{-}}
  \FunctionTok{c}\NormalTok{(}\SpecialCharTok{{-}}\DecValTok{4}\NormalTok{, }\SpecialCharTok{{-}}\DecValTok{3}\NormalTok{, }\SpecialCharTok{{-}}\DecValTok{2}\NormalTok{, }\SpecialCharTok{{-}}\DecValTok{1}\NormalTok{, }\DecValTok{0}\NormalTok{, }\DecValTok{1}\NormalTok{, }\DecValTok{2}\NormalTok{, }\DecValTok{3}\NormalTok{, }\DecValTok{4}\NormalTok{)  }\CommentTok{\# Observar que se remplazó el objeto "x" que}
                                    \CommentTok{\# se generó previamente}
\NormalTok{y }\OtherTok{\textless{}{-}}\NormalTok{ x }\SpecialCharTok{\^{}} \DecValTok{2}
\FunctionTok{plot}\NormalTok{(x, y)}
\end{Highlighting}
\end{Shaded}

\includegraphics{TutorialR_files/figure-latex/unnamed-chunk-15-1.pdf}

\begin{Shaded}
\begin{Highlighting}[]
\FunctionTok{plot}\NormalTok{(x, y, }\AttributeTok{type =} \StringTok{"b"}\NormalTok{, }\AttributeTok{col =} \StringTok{"red"}\NormalTok{)}
\end{Highlighting}
\end{Shaded}

\includegraphics{TutorialR_files/figure-latex/unnamed-chunk-15-2.pdf}

\begin{center}\rule{0.5\linewidth}{0.5pt}\end{center}

\begin{rmdtip}
\hypertarget{ejercitaciuxf3n}{%
\subsubsection{Ejercitación}\label{ejercitaciuxf3n}}
\end{rmdtip}

\begin{boxeda}
\begin{enumerate}
\def\labelenumi{\arabic{enumi}.}
\item
  Funciones y comandos básicos

  Calcule la raíz cuadrada de 10

  Calcule el perimetro del círculo de radio 5 (\(P = 2\pi \times r\))

  Calcule 270 dividido la suma entre 12 y 78

  Calcule el cuadrado de 8

  Calcule el logaritmo de 10
\item
  Asignaciones y aritmética vectorial

  Calcule el perímetro del círculo de radio 5 y guárdelo en el objeto
  \texttt{per}.

  Crear el vector de coordenadas 6,7,8,9,10 y llamarlo \texttt{z}Suma de
  dos vectores

  Calcular la suma de \texttt{z} y \texttt{x}

  Calcular el doble de \texttt{x}

  ¿Qué se obtiene haciendo el producto entre los vectores \texttt{z} y
  \texttt{x}?
\item
  R como herramienta estadística

  Generar un vector \texttt{y} con 20 realizaciones de una normal con
  media 5 y desvío estándar 2. Calcular la media y la varianza de
  \texttt{y}. Realizar un histograma.
\end{enumerate}
\end{boxeda}

\hypertarget{objetos-en-r}{%
\chapter{Objetos en R}\label{objetos-en-r}}

Los resultados de un cierto procedimiento pueden ser almacenados en diferentes
clases de objetos. R tiene cinco clases básicas de objetos, números (\texttt{numeric}),
números complejos (\texttt{complex}), cadenas de caracteres (\texttt{character}),
factores (\texttt{factor}) y valores lógicos (\texttt{logical}). Éstos pueden juntarse para
formar vectores (\texttt{vector}), matrices (\texttt{matrix}), hojas de datos (\texttt{data.frame}) o
listas (\texttt{list}). Otras clases de objetos pueden ser funciones, modelos, objetos
espaciales, entre otros. En esta sección trabajaremos con algunos de ellos.
Para conocer la clase de un objeto se utiliza la función \texttt{class}.

\hypertarget{vectores}{%
\section{Vectores}\label{vectores}}

Es el objeto más simple de R. Es importante tener en cuenta que los vectores
sólo contienen elementos de la \textbf{misma} clase básica. La función \texttt{c()} puede
utilizarse para crear vectores concatenando sus argumentos.

\begin{Shaded}
\begin{Highlighting}[]
\NormalTok{x }\OtherTok{\textless{}{-}} \FunctionTok{c}\NormalTok{(}\DecValTok{2}\NormalTok{, }\DecValTok{4}\NormalTok{, }\DecValTok{6}\NormalTok{)               }\CommentTok{\#numérico (enteros) }
\NormalTok{a }\OtherTok{\textless{}{-}} \FunctionTok{c}\NormalTok{(}\SpecialCharTok{{-}}\DecValTok{1}\NormalTok{, }\DecValTok{5}\NormalTok{, }\DecValTok{9}\NormalTok{, }\FloatTok{10.5}\NormalTok{)        }\CommentTok{\#numérico (continuo)}
\NormalTok{d }\OtherTok{\textless{}{-}} \FunctionTok{c}\NormalTok{(}\DecValTok{4}\SpecialCharTok{+}\NormalTok{2i, }\DecValTok{2}\SpecialCharTok{+}\NormalTok{5i)            }\CommentTok{\#números complejos}
\NormalTok{y }\OtherTok{\textless{}{-}} \FunctionTok{c}\NormalTok{(}\StringTok{"a"}\NormalTok{, }\StringTok{"b"}\NormalTok{, }\StringTok{"c"}\NormalTok{)         }\CommentTok{\#caracteres}
\NormalTok{z }\OtherTok{\textless{}{-}} \FunctionTok{c}\NormalTok{(}\ConstantTok{TRUE}\NormalTok{, }\ConstantTok{TRUE}\NormalTok{, }\ConstantTok{FALSE}\NormalTok{, T)  }\CommentTok{\#lógico}
\end{Highlighting}
\end{Shaded}

Si quisiéramos concatenar \texttt{x} y \texttt{a} podríamos utilizar como argumentos de la\\
función \texttt{c()}, a los objetos que quisiéramos agrupar.

\begin{Shaded}
\begin{Highlighting}[]
\NormalTok{x }\OtherTok{\textless{}{-}} \FunctionTok{c}\NormalTok{(}\DecValTok{2}\NormalTok{, }\DecValTok{4}\NormalTok{, }\DecValTok{6}\NormalTok{)}
\NormalTok{a }\OtherTok{\textless{}{-}} \FunctionTok{c}\NormalTok{(}\SpecialCharTok{{-}}\DecValTok{1}\NormalTok{, }\DecValTok{5}\NormalTok{, }\DecValTok{9}\NormalTok{, }\FloatTok{10.5}\NormalTok{)}
\NormalTok{x\_a }\OtherTok{\textless{}{-}} \FunctionTok{c}\NormalTok{(x, a)}
\NormalTok{x\_a}
\DocumentationTok{\#\# [1]  2.0  4.0  6.0 {-}1.0  5.0  9.0 10.5}
\FunctionTok{length}\NormalTok{(x\_a)}
\DocumentationTok{\#\# [1] 7}
\end{Highlighting}
\end{Shaded}

Si quiséramos comprobar que el objeto \texttt{x\_a} tiene una longitud de
7 elementos (igual a la suma de elementos de los objetos
\texttt{x} y \texttt{a}), podemos utilizar la función \texttt{length()}.

\begin{Shaded}
\begin{Highlighting}[]
\FunctionTok{length}\NormalTok{(x) }\SpecialCharTok{+} \FunctionTok{length}\NormalTok{(a)}
\DocumentationTok{\#\# [1] 7}
\FunctionTok{length}\NormalTok{(x\_a)}
\DocumentationTok{\#\# [1] 7}
\end{Highlighting}
\end{Shaded}

\begin{rmdnote}
Note que en el ejemplo anterior \texttt{T} y \texttt{F} es la forma
corta de especificar \texttt{TRUE} y \texttt{FALSE}. Es recomendable
utilizar la forma explícita de \texttt{TRUE} y \texttt{FALSE} antes que
la forma corta, dado que \texttt{T} y \texttt{F} son símbolos que pueden
redefinirse, por lo que no se debería asumir que siempre se van a
evaluar como operadores lógicos.
\end{rmdnote}

\hypertarget{secuencias}{%
\subsection{Secuencias}\label{secuencias}}

\begin{Shaded}
\begin{Highlighting}[]
\NormalTok{x }\OtherTok{\textless{}{-}} \FunctionTok{c}\NormalTok{(}\DecValTok{1}\NormalTok{, }\DecValTok{2}\NormalTok{, }\DecValTok{3}\NormalTok{, }\DecValTok{4}\NormalTok{, }\DecValTok{5}\NormalTok{)}
\NormalTok{x }\OtherTok{\textless{}{-}} \DecValTok{1}\SpecialCharTok{:}\DecValTok{10}
\NormalTok{y }\OtherTok{\textless{}{-}} \SpecialCharTok{{-}}\DecValTok{5}\SpecialCharTok{:}\DecValTok{3}
\end{Highlighting}
\end{Shaded}

Para generar secuencias de números enteros consecutivos se puede utilizar \texttt{:}, pero si se desea generar otros tipos de secuencias, por ejemplo la secuencia 4,6,8,\ldots,20, se debe utilizar la función \texttt{seq}. Los argumentos de esta función permiten generar secuencias con saltos o longitud definida por el usuario.

\begin{Shaded}
\begin{Highlighting}[]
\FunctionTok{seq}\NormalTok{(}\AttributeTok{from =} \DecValTok{4}\NormalTok{, }\AttributeTok{to =} \DecValTok{20}\NormalTok{, }\AttributeTok{by =} \DecValTok{2}\NormalTok{)}
\end{Highlighting}
\end{Shaded}

\begin{verbatim}
## [1]  4  6  8 10 12 14 16 18 20
\end{verbatim}

\begin{rmdnote}
Los argumentos de la función \texttt{seq}, permiten generar secuencia,
desde (\texttt{from} y, hasta (\texttt{to}) los valores especificados.
Se pueden especificar el incremento de cada valor (\texttt{by}), o puede
definirse el largo de la secuencia deseada (\texttt{length.out}).
\end{rmdnote}

\hypertarget{vectores-con-valores-repetidos}{%
\subsection{Vectores con valores repetidos}\label{vectores-con-valores-repetidos}}

Cuando se desea generar un vector con valores repetidos se puede utilizar la función \texttt{rep}. Esta función replica los valores que se espicifican en el primer argumento, tantas veces o hasta alcanzar la longitud total que se especifique.

\begin{Shaded}
\begin{Highlighting}[]
\FunctionTok{rep}\NormalTok{(}\DecValTok{1}\NormalTok{, }\DecValTok{5}\NormalTok{)}
\DocumentationTok{\#\# [1] 1 1 1 1 1}
\NormalTok{x }\OtherTok{\textless{}{-}} \DecValTok{1}\SpecialCharTok{:}\DecValTok{3}
\FunctionTok{rep}\NormalTok{(x, }\DecValTok{2}\NormalTok{)}
\DocumentationTok{\#\# [1] 1 2 3 1 2 3}
\FunctionTok{rep}\NormalTok{(x, }\FunctionTok{c}\NormalTok{(}\DecValTok{2}\NormalTok{,}\DecValTok{4}\NormalTok{,}\DecValTok{1}\NormalTok{)) }\CommentTok{\#En este caso repetirá el 1 dos vecesm el 2 cuatro veces y el 3 una vez.}
\DocumentationTok{\#\# [1] 1 1 2 2 2 2 3}
\FunctionTok{rep}\NormalTok{(x, }\AttributeTok{length =} \DecValTok{8}\NormalTok{)}
\DocumentationTok{\#\# [1] 1 2 3 1 2 3 1 2}
\end{Highlighting}
\end{Shaded}

\hypertarget{vectores-de-factores}{%
\subsection{Vectores de factores}\label{vectores-de-factores}}

Los vectores que se generan pueden convertirse en factores, para ello se utiliza la función \texttt{as.factor}.

\begin{Shaded}
\begin{Highlighting}[]
\NormalTok{x\_f }\OtherTok{\textless{}{-}} \FunctionTok{as.factor}\NormalTok{(x)}
\end{Highlighting}
\end{Shaded}

Tambien pueden generarse vectores que contiene factores utilizando \texttt{gl}. A esta función se le debe especificar el número de niveles del factor y el número de repeticiones. Se le puede especificar el largo del vector y las etiquetas (\texttt{labels}) de los factores.

\begin{Shaded}
\begin{Highlighting}[]
\FunctionTok{gl}\NormalTok{(}\DecValTok{3}\NormalTok{, }\DecValTok{5}\NormalTok{)}
\end{Highlighting}
\end{Shaded}

\begin{verbatim}
##  [1] 1 1 1 1 1 2 2 2 2 2 3 3 3 3 3
## Levels: 1 2 3
\end{verbatim}

\begin{Shaded}
\begin{Highlighting}[]
\FunctionTok{gl}\NormalTok{(}\DecValTok{3}\NormalTok{, }\DecValTok{5}\NormalTok{, }\AttributeTok{length =} \DecValTok{30}\NormalTok{)}
\end{Highlighting}
\end{Shaded}

\begin{verbatim}
##  [1] 1 1 1 1 1 2 2 2 2 2 3 3 3 3 3 1 1 1 1 1 2 2 2 2 2 3 3 3 3 3
## Levels: 1 2 3
\end{verbatim}

\begin{Shaded}
\begin{Highlighting}[]
\FunctionTok{gl}\NormalTok{(}\DecValTok{3}\NormalTok{, }\DecValTok{5}\NormalTok{, }\AttributeTok{labels =} \FunctionTok{c}\NormalTok{(}\StringTok{"grupo A"}\NormalTok{, }\StringTok{"grupo B"}\NormalTok{, }\StringTok{"grupo C"}\NormalTok{))}
\end{Highlighting}
\end{Shaded}

\begin{verbatim}
##  [1] grupo A grupo A grupo A grupo A grupo A grupo B grupo B grupo B grupo B
## [10] grupo B grupo C grupo C grupo C grupo C grupo C
## Levels: grupo A grupo B grupo C
\end{verbatim}

\hypertarget{seleccion-de-elementos-de-un-vector}{%
\subsection{Seleccion de elementos de un vector}\label{seleccion-de-elementos-de-un-vector}}

Los corchetes (\texttt{{[}\ {]}}) se utilizan para indicar posición de un objeto. Se utilizan del lado derecho del objeto. Dado que los vectores son elementos de una dimensión, si se desea seleccionar el primer elemento del objeto \texttt{x} se debe indicar \texttt{x{[}1{]}}.

\begin{Shaded}
\begin{Highlighting}[]
\NormalTok{x }\OtherTok{\textless{}{-}} \FunctionTok{c}\NormalTok{(}\DecValTok{3}\NormalTok{,}\DecValTok{52}\NormalTok{,}\SpecialCharTok{{-}}\DecValTok{8}\NormalTok{,}\DecValTok{2}\NormalTok{,}\DecValTok{1}\NormalTok{,}\DecValTok{7}\NormalTok{,}\DecValTok{11}\NormalTok{,}\SpecialCharTok{{-}}\DecValTok{3}\NormalTok{,}\DecValTok{0}\NormalTok{,}\DecValTok{6}\NormalTok{,}\DecValTok{23}\NormalTok{,}\DecValTok{17}\NormalTok{)}
\NormalTok{x[}\DecValTok{1}\NormalTok{]}
\end{Highlighting}
\end{Shaded}

\begin{verbatim}
## [1] 3
\end{verbatim}

\begin{Shaded}
\begin{Highlighting}[]
\NormalTok{x[}\DecValTok{3}\NormalTok{]}
\end{Highlighting}
\end{Shaded}

\begin{verbatim}
## [1] -8
\end{verbatim}

\begin{Shaded}
\begin{Highlighting}[]
\NormalTok{x[}\FunctionTok{c}\NormalTok{(}\DecValTok{1}\NormalTok{, }\DecValTok{3}\NormalTok{)]}
\end{Highlighting}
\end{Shaded}

\begin{verbatim}
## [1]  3 -8
\end{verbatim}

Si se desea sustituir un elemento del vector se puede utilizar el signo de asignación.
Por ejemplo, si se desea sustituir el tercer elemento de x por 88:

\begin{Shaded}
\begin{Highlighting}[]
\NormalTok{x[}\DecValTok{3}\NormalTok{] }\OtherTok{\textless{}{-}}  \DecValTok{88}
\NormalTok{x}
\end{Highlighting}
\end{Shaded}

\begin{verbatim}
##  [1]  3 52 88  2  1  7 11 -3  0  6 23 17
\end{verbatim}

Si se quiere obtener un vector sin algunos elementos, se debe anteponer el signo \texttt{-} al valor del índice.

\begin{Shaded}
\begin{Highlighting}[]
\NormalTok{x[}\SpecialCharTok{{-}}\DecValTok{3}\NormalTok{]}
\end{Highlighting}
\end{Shaded}

\begin{verbatim}
##  [1]  3 52  2  1  7 11 -3  0  6 23 17
\end{verbatim}

\hypertarget{matrices}{%
\section{Matrices}\label{matrices}}

Las matrices son vectores con atributo de dimensión (2 dimensiones: filas y columnas). A diferencia de los \texttt{data.frame}s, todas las columnas de las matrices son de una misma clase. Para generar matrices se puede utilizar la función \texttt{matrix}.

\begin{Shaded}
\begin{Highlighting}[]
\NormalTok{x }\OtherTok{\textless{}{-}} \DecValTok{1}\SpecialCharTok{:}\DecValTok{20}
\FunctionTok{matrix}\NormalTok{(x, }\AttributeTok{nrow =} \DecValTok{5}\NormalTok{, }\AttributeTok{ncol =} \DecValTok{4}\NormalTok{)}
\end{Highlighting}
\end{Shaded}

\begin{verbatim}
##      [,1] [,2] [,3] [,4]
## [1,]    1    6   11   16
## [2,]    2    7   12   17
## [3,]    3    8   13   18
## [4,]    4    9   14   19
## [5,]    5   10   15   20
\end{verbatim}

Las matrices pueden ser creadas uniendo filas o columnas mediante las funciones \texttt{cbind()} y \texttt{rbind()}.

\begin{Shaded}
\begin{Highlighting}[]
\NormalTok{x }\OtherTok{\textless{}{-}} \DecValTok{1}\SpecialCharTok{:}\DecValTok{3}
\NormalTok{y }\OtherTok{\textless{}{-}} \DecValTok{10}\SpecialCharTok{:}\DecValTok{12}
\FunctionTok{cbind}\NormalTok{(x, y)}
\end{Highlighting}
\end{Shaded}

\begin{verbatim}
##      x  y
## [1,] 1 10
## [2,] 2 11
## [3,] 3 12
\end{verbatim}

\begin{Shaded}
\begin{Highlighting}[]
\FunctionTok{rbind}\NormalTok{(x, y)}
\end{Highlighting}
\end{Shaded}

\begin{verbatim}
##   [,1] [,2] [,3]
## x    1    2    3
## y   10   11   12
\end{verbatim}

\hypertarget{operaciones-con-matrices}{%
\subsection{Operaciones con matrices:}\label{operaciones-con-matrices}}

\begin{itemize}
\tightlist
\item
  \texttt{A\ \%*\%\ B} : producto de matrices
\item
  \texttt{t(A)} : traspuesta de la matriz A
\item
  \texttt{solve(A)} : inversa de la matriz A
\item
  \texttt{solve(A,b)} : solución del sistema de ecuaciones Ax=b.
\item
  \texttt{svd(A)} : descomposición en valores singulares
\item
  \texttt{qr(A)} : descomposición QR
\item
  \texttt{eigen(A)} : valores y vectores propios
\item
  \texttt{diag(b)} : matriz diagonal. (b es un vector)
\item
  \texttt{diag(A)} : matriz diagonal. (A es una matriz)
\item
  \texttt{A\ \%o\%\ B\ ==\ outer(A,B)} : producto exterior de dos vectores o matrices
\end{itemize}

\hypertarget{listas}{%
\section{Listas}\label{listas}}

Una lista es la forma generalizada de un vector que puede contener elementos de diferentes clases (número, vector, matriz, lista, etc.). Para crear lista se puede utilizar la función \texttt{list()}. Dada su flexibilidad son contenedores generales de datos. Muchas funciones devuelven un conjunto de resultados de distinta longitud y distinto tipo en forma de lista.

\begin{Shaded}
\begin{Highlighting}[]
\NormalTok{n }\OtherTok{\textless{}{-}}  \FunctionTok{c}\NormalTok{(}\DecValTok{2}\NormalTok{, }\DecValTok{4}\NormalTok{, }\DecValTok{6}\NormalTok{)}
\NormalTok{s }\OtherTok{\textless{}{-}}  \FunctionTok{c}\NormalTok{(}\StringTok{"aa"}\NormalTok{, }\StringTok{"bb"}\NormalTok{, }\StringTok{"cc"}\NormalTok{, }\StringTok{"dd"}\NormalTok{, }\StringTok{"ee"}\NormalTok{)}
\NormalTok{b }\OtherTok{\textless{}{-}}  \FunctionTok{c}\NormalTok{(}\ConstantTok{TRUE}\NormalTok{, }\ConstantTok{FALSE}\NormalTok{, }\ConstantTok{TRUE}\NormalTok{, }\ConstantTok{FALSE}\NormalTok{, }\ConstantTok{FALSE}\NormalTok{)}
\NormalTok{x }\OtherTok{\textless{}{-}}  \FunctionTok{list}\NormalTok{(n, s, b)}
\NormalTok{x}
\end{Highlighting}
\end{Shaded}

\begin{verbatim}
## [[1]]
## [1] 2 4 6
## 
## [[2]]
## [1] "aa" "bb" "cc" "dd" "ee"
## 
## [[3]]
## [1]  TRUE FALSE  TRUE FALSE FALSE
\end{verbatim}

\hypertarget{hojas-de-datos-data-frames}{%
\section{Hojas de datos (Data frames)}\label{hojas-de-datos-data-frames}}

Es el objeto más común en R para almacenar datos. Sus columnas pueden ser de diferentes clases por ejemplo variables continuas y categóricas. Este tipo de objetos puede generarse mediante la función \texttt{data.frame()}.

\begin{rmdnote}
\texttt{data.frame} convierte los vectores de caracteres en factores
automáticamente.
\end{rmdnote}

\begin{Shaded}
\begin{Highlighting}[]
\NormalTok{x1 }\OtherTok{\textless{}{-}} \DecValTok{1}\SpecialCharTok{:}\DecValTok{10}
\NormalTok{x2 }\OtherTok{\textless{}{-}} \DecValTok{24}\SpecialCharTok{:}\DecValTok{33}
\NormalTok{x3 }\OtherTok{\textless{}{-}} \FunctionTok{gl}\NormalTok{(}\DecValTok{2}\NormalTok{, }\DecValTok{5}\NormalTok{, }\AttributeTok{labels =} \FunctionTok{c}\NormalTok{(}\StringTok{"si"}\NormalTok{,}\StringTok{"no"}\NormalTok{))}
\NormalTok{x4 }\OtherTok{\textless{}{-}}\NormalTok{ letters[}\DecValTok{1}\SpecialCharTok{:}\DecValTok{10}\NormalTok{]}
\FunctionTok{data.frame}\NormalTok{(}\AttributeTok{A =}\NormalTok{ x1, }\AttributeTok{B =}\NormalTok{ x2, }\AttributeTok{C =}\NormalTok{ x3, }\AttributeTok{D =}\NormalTok{ x4)}
\end{Highlighting}
\end{Shaded}

\begin{verbatim}
##     A  B  C D
## 1   1 24 si a
## 2   2 25 si b
## 3   3 26 si c
## 4   4 27 si d
## 5   5 28 si e
## 6   6 29 no f
## 7   7 30 no g
## 8   8 31 no h
## 9   9 32 no i
## 10 10 33 no j
\end{verbatim}

\hypertarget{algunas-funciones-buxe1sicas-predefinidas}{%
\section{Algunas funciones básicas predefinidas}\label{algunas-funciones-buxe1sicas-predefinidas}}

\begin{itemize}
\tightlist
\item
  \texttt{summay()}
\item
  \texttt{mean()}
\item
  \texttt{var()}
\item
  \texttt{sd()}
\item
  \texttt{cor()}
\item
  \texttt{sum()}
\item
  \texttt{min()}
\item
  \texttt{max()}
\item
  \texttt{seq()}
\item
  \texttt{which()}

  \begin{itemize}
  \tightlist
  \item
    \texttt{which.min()}
  \item
    \texttt{which.min()}
  \end{itemize}
\item
  \texttt{length()}
\item
  \texttt{table()}
\item
  \texttt{is.na()}
\item
  \texttt{is.null()}
\item
  \texttt{complete.cases()}
\item
  \texttt{as.character()}
\item
  \texttt{as.numeric()}
\item
  \texttt{paste()}
\item
  \texttt{gsub()}
\item
  \texttt{unique()}
\item
  \texttt{lm()}
\item
  \texttt{dim()}
\item
  \texttt{nrow()}
\item
  \texttt{ncol()}
\item
  \texttt{colnames()}
\item
  \texttt{rownames()}
\item
  \texttt{edit()}
\item
  \texttt{cbind()}
\item
  \texttt{rbind()}
\item
  \texttt{order()}
\item
  \texttt{install.packages()}
\item
  \texttt{library()}
\end{itemize}

\begin{rmdtip}
\hypertarget{ejercicio-objetos}{%
\subsection{Ejercitación}\label{ejercicio-objetos}}
\end{rmdtip}

\begin{enumerate}
\def\labelenumi{\arabic{enumi}.}
\item
  Vectores

  Genere un vector \texttt{b} el cual contenga los valores de \texttt{x} y \texttt{a} ¿Cuantos elementos tiene el vector \texttt{b}?
\item
  Secuencias

  Genere la secuencia 8,7,6,5,4

  seq(4,20,2) ¿Este comando da eror? ¿Por qué?

  Genere usando comandos para secuencias el vector de componentes: 1, 2, 3, 4, 5, 6, 73, 72, 71, 70, 69, 68, 3, 6, 9, 12, 15, 18.
\item
  Repetir valores

  Genere un vector de componentes ``azul'', ``azul'',``azul'', ``azul'', ``amarillo'', ``amarillo'', ``verde'', ``verde'',``verde'', llamado \texttt{col}. ¿Es un vector de factores?
\item
  Matrices

  Calcule la inversa y los autovalores y autovectores de \texttt{A\ \ =\ \ matrix(c(1,3,3,9,5,9,3,5,6),\ nrow\ =\ 3)}
\end{enumerate}

\hypertarget{control-de-flujo}{%
\chapter{Control de flujo}\label{control-de-flujo}}

\hypertarget{construcciuxf3n-condicional-if}{%
\section{\texorpdfstring{Construcción condicional \texttt{if}}{Construcción condicional if}}\label{construcciuxf3n-condicional-if}}

Es de la forma \texttt{if\ (expr\ 1)\ expr\ 2\ else\ expr\ 3} donde \texttt{expr\ 1} debe producir un valor logico. Si \texttt{expr\ 1} es verdadero (\texttt{T}), se ejecutara \texttt{expr\ 2}. Si \texttt{expr\ 1} es falso (\texttt{F}), y se ha escrito la opcion else, que es opcional, se ejecutara \texttt{expr\ 3}.

\begin{Shaded}
\begin{Highlighting}[]
\ControlFlowTok{if}\NormalTok{( }\DecValTok{3} \SpecialCharTok{\textgreater{}} \DecValTok{2}\NormalTok{) }\FunctionTok{print}\NormalTok{(}\StringTok{"yes"}\NormalTok{)}
\end{Highlighting}
\end{Shaded}

\begin{verbatim}
## [1] "yes"
\end{verbatim}

\begin{Shaded}
\begin{Highlighting}[]
\ControlFlowTok{if}\NormalTok{( }\DecValTok{2} \SpecialCharTok{\textgreater{}} \DecValTok{3}\NormalTok{) }\FunctionTok{print}\NormalTok{(}\StringTok{"yes"}\NormalTok{)}
\ControlFlowTok{if}\NormalTok{( }\DecValTok{2} \SpecialCharTok{\textgreater{}} \DecValTok{3}\NormalTok{) }\FunctionTok{print}\NormalTok{(}\StringTok{"yes"}\NormalTok{) }\ControlFlowTok{else} \FunctionTok{print}\NormalTok{(}\StringTok{"no"}\NormalTok{)}
\end{Highlighting}
\end{Shaded}

\begin{verbatim}
## [1] "no"
\end{verbatim}

Ejemplo con dos condiciones
supongamos que \texttt{x\ \textless{}-\ 75} es la nota numerica de examen de un alumno, queremos asignar nota ``A'', ``B'' o ``C''

\begin{Shaded}
\begin{Highlighting}[]
\ControlFlowTok{if}\NormalTok{(x }\SpecialCharTok{\textless{}} \DecValTok{60}\NormalTok{) nota }\OtherTok{=} \StringTok{"C"}
\ControlFlowTok{if}\NormalTok{(x }\SpecialCharTok{\textgreater{}=} \DecValTok{60} \SpecialCharTok{\&}\NormalTok{ x }\SpecialCharTok{\textless{}} \DecValTok{80}\NormalTok{) nota }\OtherTok{=} \StringTok{"B"}
\ControlFlowTok{if}\NormalTok{(x }\SpecialCharTok{\textgreater{}=}  \DecValTok{80}\NormalTok{) nota }\OtherTok{=} \StringTok{"A"}
\end{Highlighting}
\end{Shaded}

\texttt{ifelse} es la versión vectorizada de \texttt{if}

Ejemplo

\begin{Shaded}
\begin{Highlighting}[]
\NormalTok{nota.num }\OtherTok{\textless{}{-}} \FunctionTok{c}\NormalTok{(}\DecValTok{39}\NormalTok{, }\DecValTok{51}\NormalTok{, }\DecValTok{60}\NormalTok{, }\DecValTok{65}\NormalTok{, }\DecValTok{72}\NormalTok{, }\DecValTok{78}\NormalTok{, }\DecValTok{79}\NormalTok{, }\DecValTok{83}\NormalTok{, }\DecValTok{85}\NormalTok{, }\DecValTok{85}\NormalTok{, }\DecValTok{87}\NormalTok{, }\DecValTok{89}\NormalTok{, }\DecValTok{91}\NormalTok{, }\DecValTok{95}\NormalTok{, }\DecValTok{96}\NormalTok{, }\DecValTok{97}\NormalTok{, }\DecValTok{100}\NormalTok{, }\DecValTok{100}\NormalTok{)}

\NormalTok{prueba }\OtherTok{\textless{}{-}} \FunctionTok{ifelse}\NormalTok{ (nota.num }\SpecialCharTok{\textgreater{}=} \DecValTok{60}\NormalTok{, }\StringTok{"aprobado"}\NormalTok{, }\StringTok{"desaprobado"}\NormalTok{)}
\NormalTok{prueba}
\end{Highlighting}
\end{Shaded}

\begin{verbatim}
##  [1] "desaprobado" "desaprobado" "aprobado"    "aprobado"    "aprobado"   
##  [6] "aprobado"    "aprobado"    "aprobado"    "aprobado"    "aprobado"   
## [11] "aprobado"    "aprobado"    "aprobado"    "aprobado"    "aprobado"   
## [16] "aprobado"    "aprobado"    "aprobado"
\end{verbatim}

\hypertarget{construcciuxf3n-repetitiva-for}{%
\section{\texorpdfstring{Construcción repetitiva \texttt{for}}{Construcción repetitiva for}}\label{construcciuxf3n-repetitiva-for}}

Es de la forma \texttt{for\ (nombre\ in\ expr\ 1)\ expr\ 2} donde \texttt{nombre} es la variable de control de iteración, \texttt{expr\ 1} es un vector (a menudo de la forma \texttt{m:n}), y \texttt{expr\ 2} es una expresión, a menudo agrupada, en cuyas sub-expresiones puede aparecer la variable de control, \texttt{nombre}. \texttt{expr\ 2} se evalua repetidamente conforme \texttt{nombre} recorre los valores del vector \texttt{expr\ 1}.

\begin{Shaded}
\begin{Highlighting}[]
\ControlFlowTok{for}\NormalTok{ (i }\ControlFlowTok{in} \DecValTok{1}\SpecialCharTok{:}\DecValTok{10}\NormalTok{) }\FunctionTok{print}\NormalTok{(i)}
\end{Highlighting}
\end{Shaded}

\begin{verbatim}
## [1] 1
## [1] 2
## [1] 3
## [1] 4
## [1] 5
## [1] 6
## [1] 7
## [1] 8
## [1] 9
## [1] 10
\end{verbatim}

\begin{Shaded}
\begin{Highlighting}[]
\NormalTok{x }\OtherTok{=} \FunctionTok{numeric}\NormalTok{(}\DecValTok{10}\NormalTok{)}
\ControlFlowTok{for}\NormalTok{ (i }\ControlFlowTok{in} \DecValTok{1}\SpecialCharTok{:}\DecValTok{10}\NormalTok{) x[i] }\OtherTok{=}\NormalTok{ i}\SpecialCharTok{\^{}}\DecValTok{2}

\NormalTok{y }\OtherTok{=} \DecValTok{0}
\ControlFlowTok{for}\NormalTok{ (i }\ControlFlowTok{in} \DecValTok{1}\SpecialCharTok{:}\DecValTok{10}\NormalTok{) y }\OtherTok{=}\NormalTok{ y }\SpecialCharTok{+}\NormalTok{ i}
\end{Highlighting}
\end{Shaded}

\hypertarget{construccion-repetitiva-while}{%
\section{\texorpdfstring{Construccion repetitiva \texttt{while}}{Construccion repetitiva while}}\label{construccion-repetitiva-while}}

Es de la forma \texttt{while\ (expr1)\ expr2}, indicando que se quiere repetir la acción \texttt{expr2} mientras que ocurra \texttt{expr1}.

\begin{Shaded}
\begin{Highlighting}[]
\NormalTok{i }\OtherTok{=} \DecValTok{0}
\ControlFlowTok{while}\NormalTok{ (i }\SpecialCharTok{\textless{}} \DecValTok{15}\NormalTok{) \{}\FunctionTok{print}\NormalTok{(i); i }\OtherTok{=}\NormalTok{ i}\SpecialCharTok{+}\DecValTok{1}\NormalTok{\}}
\end{Highlighting}
\end{Shaded}

\begin{verbatim}
## [1] 0
## [1] 1
## [1] 2
## [1] 3
## [1] 4
## [1] 5
## [1] 6
## [1] 7
## [1] 8
## [1] 9
## [1] 10
## [1] 11
## [1] 12
## [1] 13
## [1] 14
\end{verbatim}

\begin{center}\rule{0.5\linewidth}{0.5pt}\end{center}

\begin{rmdtip}
\hypertarget{ejercitaciuxf3n}{%
\subsection{Ejercitación}\label{ejercitaciuxf3n}}
\end{rmdtip}

\begin{enumerate}
\def\labelenumi{\arabic{enumi}.}
\item
  Construcción condicional

  Si se quiere poner notas ``A'', ``B'' o ``C'': ``C'' si final\_score \textless60, ``B'' si 60 =\textless{} final\_score \textless{} 80, ``A'' si 80 =\textless{} final\_score =\textless{} 100.
\item
  Construcción repetitiva

  Usar un ciclo \texttt{for} para contar la cantidad de números mayores a 10 en el vector \texttt{x\ \textless{}-\ c(2,5,3,9,8,11,6,8,12,3,57,56)}
\end{enumerate}

\hypertarget{generar-nuevas-funciones}{%
\chapter{Generar nuevas funciones}\label{generar-nuevas-funciones}}

R es un lenguaje que permite crear nuevas funciones. Una funcion se define con una asignacion de la forma:

\begin{Shaded}
\begin{Highlighting}[]
\NormalTok{nombre }\OtherTok{\textless{}{-}} \ControlFlowTok{function}\NormalTok{(arg1, arg2, ...) \{}
\NormalTok{   expresion}
\NormalTok{ \}}
\end{Highlighting}
\end{Shaded}

La \texttt{expresion} es una fórmula o grupo de formulas (o sentencias) que utilizan los argumentos para calcular uno o varios valores. El resultado de dicha expresión es el valor que proporciona R en su salida y este puede ser un número, un vector, un gráfico, una lista y/o un mensaje. Una función devuelve el último valor impreso en la consola.

Ejemplos:

\begin{Shaded}
\begin{Highlighting}[]
\NormalTok{funcion1 }\OtherTok{\textless{}{-}} \ControlFlowTok{function}\NormalTok{(x)\{ y }\OtherTok{=}\NormalTok{ x }\SpecialCharTok{+} \DecValTok{4}\NormalTok{\}}

\NormalTok{(a}\OtherTok{\textless{}{-}}\FunctionTok{funcion1}\NormalTok{(}\DecValTok{5}\NormalTok{))}
\end{Highlighting}
\end{Shaded}

\begin{verbatim}
## [1] 9
\end{verbatim}

En en caso siguiente, si se desea guardar el resultado en un objeto solo se guardará el rango (último valor impreso en consola).

\begin{Shaded}
\begin{Highlighting}[]
\NormalTok{funcion2 }\OtherTok{\textless{}{-}} \ControlFlowTok{function}\NormalTok{(muestra)\{     }\CommentTok{\#El único argumento es un vector de datos}
\NormalTok{  media }\OtherTok{=} \FunctionTok{mean}\NormalTok{(muestra, }\AttributeTok{na.rm =}\NormalTok{ T)}
\NormalTok{  varianza }\OtherTok{=} \FunctionTok{var}\NormalTok{(muestra, }\AttributeTok{na.rm =}\NormalTok{ T)}
\NormalTok{  rango }\OtherTok{=} \FunctionTok{max}\NormalTok{(muestra, }\AttributeTok{na.rm =}\NormalTok{ T) }\SpecialCharTok{{-}} \FunctionTok{min}\NormalTok{(muestra, }\AttributeTok{na.rm =}\NormalTok{ T)}
  \FunctionTok{print}\NormalTok{(media)}
  \FunctionTok{print}\NormalTok{(varianza)}
  \FunctionTok{print}\NormalTok{(rango)}

\NormalTok{\}}

\FunctionTok{funcion2}\NormalTok{(}\FunctionTok{rnorm}\NormalTok{(}\DecValTok{40}\NormalTok{,}\DecValTok{5}\NormalTok{,}\DecValTok{16}\NormalTok{))}
\end{Highlighting}
\end{Shaded}

\begin{verbatim}
## [1] 8.674483
## [1] 223.9998
## [1] 59.92459
\end{verbatim}

Para que guarde los tres resultados hay que especificar que se haga una lista o vector.

\begin{Shaded}
\begin{Highlighting}[]
\NormalTok{funcion3 }\OtherTok{\textless{}{-}} \ControlFlowTok{function}\NormalTok{(muestra)\{     }
\NormalTok{  med }\OtherTok{=} \FunctionTok{mean}\NormalTok{(muestra, }\AttributeTok{na.rm =}\NormalTok{ T)}
\NormalTok{  vari }\OtherTok{=} \FunctionTok{var}\NormalTok{(muestra, }\AttributeTok{na.rm =}\NormalTok{ T)}
\NormalTok{  rang }\OtherTok{=} \FunctionTok{max}\NormalTok{(muestra, }\AttributeTok{na.rm =}\NormalTok{ T) }\SpecialCharTok{{-}} \FunctionTok{min}\NormalTok{(muestra, }\AttributeTok{na.rm =}\NormalTok{ T)}
  \CommentTok{\# list(media = med, varianza = vari ,rango = rang)}
  \FunctionTok{c}\NormalTok{(}\StringTok{"Media"}\OtherTok{=}\NormalTok{med,}\StringTok{"Var"}\OtherTok{=}\NormalTok{vari,}\StringTok{"Rango"}\OtherTok{=}\NormalTok{rang)}
\NormalTok{\}}

\NormalTok{ej }\OtherTok{\textless{}{-}} \FunctionTok{funcion3}\NormalTok{(}\DecValTok{1}\SpecialCharTok{:}\DecValTok{20}\NormalTok{)}
\NormalTok{ej}
\end{Highlighting}
\end{Shaded}

\begin{verbatim}
## Media   Var Rango 
##  10.5  35.0  19.0
\end{verbatim}

Los diferentes argumentos de las funciones se separan con \texttt{,}. Éstos pueden tener un valor por defecto. Para especificarlo, en el momento de crear la función se especifica con el signo \texttt{=}, cuál es el valor que se usará si el usuario no lo especifica explícitamente.

\begin{Shaded}
\begin{Highlighting}[]
\NormalTok{funcion4 }\OtherTok{\textless{}{-}} \ControlFlowTok{function}\NormalTok{(a,b,}\AttributeTok{c =} \DecValTok{4}\NormalTok{,}\AttributeTok{d =} \ConstantTok{FALSE}\NormalTok{)\{}
  \ControlFlowTok{if}\NormalTok{ (d }\SpecialCharTok{==} \ConstantTok{FALSE}\NormalTok{) x1 }\OtherTok{\textless{}{-}}\NormalTok{ a}\SpecialCharTok{*}\NormalTok{b }\ControlFlowTok{else}\NormalTok{ x1 }\OtherTok{\textless{}{-}}\NormalTok{ a}\SpecialCharTok{*}\NormalTok{b }\SpecialCharTok{+}\NormalTok{ c}
\NormalTok{  x1}
\NormalTok{\}}
\end{Highlighting}
\end{Shaded}

\begin{center}\rule{0.5\linewidth}{0.5pt}\end{center}

\begin{rmdtip}
\hypertarget{ejercitaciuxf3n}{%
\subsection{Ejercitación}\label{ejercitaciuxf3n}}
\end{rmdtip}

\begin{enumerate}
\def\labelenumi{\arabic{enumi}.}
\item
  Funciones

  Genere una función que grafique una variable en función de otra y coloque nombre al eje x que por defecto sea: ``mi eje x''
\end{enumerate}

  \bibliography{book.bib,packages.bib}

\end{document}
